The \scommand{move} command can be used to place a player directly
onto a desired position on the field. \scommand{move} exists to set up
the team and does not work during normal play. It is available at the
beginning of each half (play mode \PM{before\_kick\_off}) and after a
goal has been scored (play modes \PM{goal\_r\_\emph{n}} or
\PM{goal\_l\_\emph{n}}). In these situations, players can be placed on
any position in their own half (i.e. $X < 0$) and can be moved any
number of times, as long as the play mode does not change. Players
moved to a position on the opponent half will be set to a random
position on their own side by the server.

A second purpose of the \scommand{move} command is to move the goalie
within the penalty area after the goalie caught the ball.
If the goalie caught the ball, it can move
together with the ball within the penalty area. The goalie is
allowed to move \sparam{goalie\_max\_moves} times before it kicks the
ball. Additional \scommand{move} commands do not have any
effect and the server will respond with
\texttt{(error~too\_many\_moves)}. 

\begin{table}[htbp]
  \begin{center}
    \begin{tabular}[h]{|l|r|}
      \hline
      \textbf{Parameter in \file{server.conf}} & \textbf{Value} \\\hline
      \sparam{goalie\_max\_moves} & 2 \\\hline
    \end{tabular}
    \caption{Parameter for the move command}
    \label{tab:move}
  \end{center}
\end{table}

%%% Local Variables: 
%%% mode: latex
%%% TeX-master: "manual"
%%% End: 
